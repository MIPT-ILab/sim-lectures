% Compile with XeLaTeX, TeXLive 2013 or more recent
\documentclass{beamer}

% Base packages
\usepackage{fontspec}

\usepackage{xunicode}
\usepackage{xltxtra}

\usepackage{amsfonts}
\usepackage{amsmath}
\usepackage{longtable}
\usepackage{csquotes}
\usepackage{standalone}

\usepackage{graphicx}
\graphicspath{{./images/}}

\usepackage{tikz}
\usetikzlibrary{arrows,decorations.pathmorphing,backgrounds,positioning,fit,petri}

\usepackage{listings}
\lstset{language=C, basicstyle=\ttfamily, breaklines=true, keepspaces=true, keywordstyle=\color{blue}}

% Setup Russian hyphenation
\usepackage{polyglossia}
\setdefaultlanguage[spelling=modern]{russian} % for polyglossia
\setotherlanguage{english} % for polyglossia

% Setup fonts
\newfontfamily\russianfont{CMU Serif}
\setromanfont{CMU Serif}
\setsansfont{CMU Sans Serif}
\setmonofont{CMU Typewriter Text}

% Be able to insert hyperlinks
\usepackage{hyperref}
\hypersetup{colorlinks=true, linkcolor=black, filecolor=black, citecolor=black, urlcolor=blue , pdfauthor=Grigory Rechistov <grigory.rechistov@phystech.edu> and Evgeny Yulyugin <yulyugin@gmail.com>, pdftitle=Программное моделирование вычислительнызх систем}
% \usepackage{url}

% Misc optional packages
\usepackage{underscore}
\usepackage{amsthm}

% A new command to mark not done places
\newcommand{\todo}[1][]{{\color{red}TODO\ #1}}

\newcommand{\abbr}{\textit{англ.}\ }

\subtitle{Курс «Программное моделирование вычислительных систем»}
\subject{Лекция}
\date{\today}
\pgfdeclareimage[height=0.5cm]{ilab-logo}{../common/ilab-noletters.png}
\logo{\pgfuseimage{ilab-logo}}

\typeout{Copyright 2014 Grigory Rechistov and Evgeny Yulyugin}

\usetheme{Berlin}
\setbeamertemplate{navigation symbols}{}%remove navigation symbols


\title{Языки разработки моделей и аппаратуры}
%\author[Григорий Речистов]{Григорий Речистов \\ \small{\href{mailto:grigory.rechistov@phystech.edu}{grigory.rechistov@phystech.edu}}}
\author[Евгений Юлюгин]{Евгений Юлюгин \\ \small{\href{mailto:yulyugin@gmail.com}{yulyugin@gmail.com}}}

\begin{document}

\begin{frame}
\titlepage
\end{frame}

\section{Обзор}

\begin{frame}
\tableofcontents
\end{frame} 

\begin{frame}{На прошлой лекции}

\begin{itemize}
  \item Аналитические модели,
  \item Трассы (оффлайн модели)
\end{itemize}

\end{frame}

\section{Компоненты симулятора}

\begin{frame}

\begin{figure}[htp]
    \centering
    \inputpicture{breakdown}
\end{figure}

\end{frame}

\section{Яызки разработки моделей}

\begin{frame}{Вопрос}

На каком языке программирования должен быть написан симулятор?

\end{frame}

\begin{frame}{Использование языков общего назначения (C, C++, ...)}

\begin{itemize}
  \item Используется ООП
  \item Написание моделей <<с нуля>>
  \item Особенности языков
  \begin{itemize}
    \item Спацифика int (int32\_t, int64\_t)
    \item thread safety
    \item malloc/free
  \end{itemize}
\end{itemize}

\end{frame}

\begin{frame}{Недостаток}

Низкая скорость разработки!

\end{frame}

\begin{frame}{Решение}

\begin{itemize}
  \item Создание библиотек, реализующих общие примитивы моделирования (SystemC/TLM).
  \item Использование специальных языков написания моделей (DML).
\end{itemize}

\end{frame}

\begin{frame}{Сборка модели SystemC}

\begin{figure}[htp]
    \centering
    \inputpicture{systemc-compilation}
\end{figure}

\end{frame}

\begin{frame}[fragile]{Пример SystemC кода}

\begin{lstlisting}[basicstyle=\footnotesize]
#include "systemc.h"

SC_MODULE(adder) {        // module (class) declaration
  sc_in<int> a, b;        // ports
  sc_out<int> sum;
  void do_add() {         // process
    sum.write(a.read() + b.read()); //or just sum = a + b
  }
  SC_CTOR(adder) {        // constructor
    SC_METHOD(do_add);    // register do_add to kernel
    sensitive << a << b;  // sensitivity list of do_add
  }
};
\end{lstlisting}

\end{frame}

\begin{frame}{Сборка DML}

\begin{figure}[htp]
    \centering
    \inputpicture{dml-compilation}
\end{figure}

\end{frame}

\begin{frame}[fragile]{Пример кода DML}

\begin{lstlisting}
dml 1.2;

device simple_dml_device;

parameter desc = “Simple DML device”;
parameter documentation = “This is an implementation of simple DML device.”;

bank regs {
  parameter register_size = 4;

  register r1 @ 0x1000;
  register r2 @ 0x1004;
}
\end{lstlisting}

\end{frame}


\begin{frame}{Абстракции аппаратуры}

\begin{itemize}
    \item Сигналы --- логический уровень (0 или 1),
    \item Шины --- передача группы бит,
    \item Операции над отдельными битами,
    \item Транзакции --- отображение направления сигнала.
\end{itemize}

\end{frame}

\begin{frame}{Абстракции аппаратуры}

\begin{itemize}
    \item Расширенный значения для уровней сигналов: непроводяций (hi-Z) или неопределенный (X),
    \item Абстракции хранения данных: группы регистров, банки памяти...
    \begin{itemize}
        \item Побочные эффекты.
    \end{itemize}
\end{itemize}

\end{frame}

\begin{frame}{Абстакции аппаратуры}

\begin{itemize}
    \item Карты памяти,
    \item Задержки событий --- разные для различных действий.
\end{itemize}

\end{frame}

\section{Разработка процессора}

\begin{frame}{Проблемы}

Инструменты и документы, успользуемые при создании процессора:

\begin{itemize}
    \item Функциональный симулятор набора инструкций,
    \item Точная потактовая модель,
    \item Дизассемблер машинного кода,
    \item Компилятор с языка высого уровня в ассемблерный код,
    \item Документация к аппаратуре
    \item Синтезируемое описание новой архитектуры.
\end{itemize}

\end{frame}

\begin{frame}{Языки разработки набора команд}

\begin{columns}[t]
    \begin{column}[T]{0.7\textwidth}
    \begin{figure}
        \centering
        \inputpicture{lisa}
    \end{figure}
    \end{column}

    \begin{column}[T]{0.3\textwidth}
        Примеры:
        \begin{itemize}
            \item LISA,
            \item ISDL,
            \item SimGen.
        \end{itemize}
    \end{column}
\end{columns}

\end{frame}

\begin{frame}{Преимущества и недостатки}

\begin{columns}[t]
    \begin{column}[T]{0.5\textwidth}
    Недостатки:
    \begin{itemize}
        \item Генерируется не самый быстрый код,
        \item Код может быть не компактен,
        \item Модель может работать медленне.
    \end{itemize}
    \end{column}
    \begin{column}[T]{0.5\textwidth}
    Преимущества:
    \begin{itemize}
        \item Скорость создания/модификации,
        \item Согласованность.
    \end{itemize}
    \end{column}
\end{columns}

\end{frame}

\begin{frame}{Verilog}

Создатели Phil Moorby \& Prabhu Goel, работавшие в «Automated Integrated Design Systems», 1984 г.

Позволяет генерировать netlist --- логически эквивалентное описание, состоящее из элементарных логических примитивов.

Команды делятся на два типа:

\begin{itemize}
    \item Синтезируемые --- представленные в аппаратуре,
    \item Несинтезируемые --- для отладки и симуляции.
\end{itemize}

\end{frame}

\begin{frame}{VHDL}

\begin{itemize}
    \item Был разработан в 1983 г. по заказу Министерства обороны США,
    \item Первоначально предназначался для моделирования, но позже появилась и синтезируемое подмножество.
\end{itemize}

\end{frame}

% Use [fragile] option to insert listings
% \begin{frame}[fragile]

\section{Конец}
% The final "thank you" frame 

\begin{frame}{На следующей лекции}

Итоговая контрольная работа.

\end{frame}

\begin{frame}

{\huge{Спасибо за внимание!}\par}

\vfill

\tiny{\textit{Замечание}: все торговые марки и логотипы, использованные в данном материале, являются собственностью их владельцев. Представленная здесь точка зрения отражает личное мнение автора, не выступающего от лица какой-либо организации.}

\end{frame}

\end{document}
