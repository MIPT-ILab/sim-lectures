% Compile with XeLaTeX, TeXLive 2013 or more recent
\documentclass{beamer}

% Base packages
\usepackage{fontspec}

\usepackage{xunicode}
\usepackage{xltxtra}

\usepackage{amsfonts}
\usepackage{amsmath}
\usepackage{longtable}
\usepackage{csquotes}
\usepackage{standalone}

\usepackage{graphicx}
\graphicspath{{./images/}}

\usepackage{tikz}
\usetikzlibrary{arrows,decorations.pathmorphing,backgrounds,positioning,fit,petri}

\usepackage{listings}
\lstset{language=C, basicstyle=\ttfamily, breaklines=true, keepspaces=true, keywordstyle=\color{blue}}

% Setup Russian hyphenation
\usepackage{polyglossia}
\setdefaultlanguage[spelling=modern]{russian} % for polyglossia
\setotherlanguage{english} % for polyglossia

% Setup fonts
\newfontfamily\russianfont{CMU Serif}
\setromanfont{CMU Serif}
\setsansfont{CMU Sans Serif}
\setmonofont{CMU Typewriter Text}

% Be able to insert hyperlinks
\usepackage{hyperref}
\hypersetup{colorlinks=true, linkcolor=black, filecolor=black, citecolor=black, urlcolor=blue , pdfauthor=Grigory Rechistov <grigory.rechistov@phystech.edu> and Evgeny Yulyugin <yulyugin@gmail.com>, pdftitle=Программное моделирование вычислительнызх систем}
% \usepackage{url}

% Misc optional packages
\usepackage{underscore}
\usepackage{amsthm}

% A new command to mark not done places
\newcommand{\todo}[1][]{{\color{red}TODO\ #1}}

\newcommand{\abbr}{\textit{англ.}\ }

\subtitle{Курс «Программное моделирование вычислительных систем»}
\subject{Лекция}
\date{\today}
\pgfdeclareimage[height=0.5cm]{ilab-logo}{../common/ilab-noletters.png}
\logo{\pgfuseimage{ilab-logo}}

\typeout{Copyright 2014 Grigory Rechistov and Evgeny Yulyugin}

\usetheme{Berlin}
\setbeamertemplate{navigation symbols}{}%remove navigation symbols


\title{Моделирование центрального процессора с помощью интерпретации}
\author[Евгений Юлюгин]{Евгений Юлюгин \\ \small{\href{mailto:yulyugin@gmail.com}{yulyugin@gmail.com}}}

\begin{document}

\begin{frame}
\titlepage
\end{frame}

\section*{Обзор}

\begin{frame}{На прошлой лекции}
\end{frame}

\begin{frame}{На этой лекции}
\tableofcontents
\end{frame} 

% Use [fragile] option to insert listings
% \begin{frame}[fragile]

\section{Цикл работы процессора}

\begin{frame}{Цикл работы процессора}
\centering
\inputpicture{interp-cycle-expanded}
\end{frame}

\begin{frame}[fragile]{Переключаемый интерпретатор (switched)}
\begin{lstlisting}
while (run) {
    raw_code = fetch(PC);
    (opcode, operands) = decode(raw_code);
    switch (opcode) {

    case opcode1:
        func1(operands); PC++; break;

    case opcode2:
        func2(operands); PC++; break;

    /*...*/
    }
}
\end{lstlisting}
\end{frame}

\begin{frame}{Уточненный цикл работы}
\centering
\resizebox{9cm}{7cm}{
\inputpicture{interp-cycle-expanded-exception}
}
\end{frame}

\section{Fetch}

\begin{frame}{Порядок байт при доступах}

\begin{itemize}
    \item Порядок от младшего к старшему (\abbr little-endian);
    \item Порядок от старшего к младшему (\abbr big-endian);
    \item Смешанный порядок (\abbr middle-endian).
\end{itemize}

\pause

\begin{table}[htpb]
    \centering
    \begin{tabular}{|l|l|}
    \hline
    Представление   &   D4 + C3 * 100 + B2 * 10000 + A1 * 1000000   \\
    \hline
    Little-endian   &   D4, C3, B2, A1                              \\
    \hline
    Big-endian      &   A1, B2, C3, D4                              \\
    \hline
    \end{tabular}
\end{table}

\end{frame}

\begin{frame}{Бит, байт, слово}

Бит \pause --- наименьшая единица информации.

\bigskip
\pause

Байт \pause --- минимальная адресуемая (в данной архитектуре) единица хранения
информации.

\bigskip
\pause

Октет --- восемь бит.

\bigskip
\pause

Машинное слово \pause --- максимальный объём информации, который ЦПУ может
обработать единовременно.

\bigskip
\pause

Intel: word — 16 бит, dword — 32 бит, qword — 64 бит.

\end{frame}

\section{Decode}

\begin{frame}{Анатомия инструкции}
\centering
\inputpicture{instruction-anatomy}
\end{frame}

\section{Execute}

\section{Write Back}

\section{Advance PC}

\section*{Конец}
% The final "thank you" frame 

\begin{frame}{На следующей лекции}
\end{frame}

\begin{frame}

{\huge{Спасибо за внимание!}\par}

\vfill

\tiny{\textit{Замечание}: все торговые марки и логотипы, использованные в данном материале, являются собственностью их владельцев. Представленная здесь точка зрения отражает личное мнение автора, не выступающего от лица какой-либо организации.}

\end{frame}

\end{document}
